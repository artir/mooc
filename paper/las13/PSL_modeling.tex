%!TEX root = nips2013.tex

\section{Modeling Learner Engagement in PSL}
\label{others}

In this section, we  provide a brief description of PSL. Additional details about PSL are available in \cite{psl}. We also discuss the various features (behavioral, linguistic and structural), which learners exhibit online while attending these courses. We then describe the PSL models to model student behavior which capture these features. We begin with a simple model which predicts performance using only behavioral, linguistic, structural and temporal features and then increase its complexity by including forms of engagement and disengagement as a latent feature.

\subsection{Probabilistic Soft Logic}
\label{sec:psl}

Probabilistic soft logic (PSL) \cite{psl} is a framework for collective, probabilistic reasoning
in relational domains (e.g., personalized medicine and opinion diffusion). PSL uses syntax based on first-order logic as a templating language for continuous graphical models over random variables representing soft truth values.
Like other statistical relational learning methods \cite{srl-book}, PSL uses weighted rules to model the dependencies in a domain. However, one distinguishing aspect is that PSL uses continuous variables to represent truth values, relaxing Boolean truth values to the interval [0,1]. Triangular norms, which are continuous relaxations of logical connectives AND
and OR, are used to combine the atoms in the first-order clauses. 
As a result of the soft formulation and the triangular norms, the underlying probabilistic model is a \emph{hinge-loss Markov random field} HL-MRF \cite{bach-uai13}.  
Inference in HL-MRFs is a convex optimization problem, which makes working with PSL very efficient in comparison to relational modeling tools that use discrete representations.

HL-MRFs admit various learning algorithms for fully-supervised training data, and are amenable to point-estimate ``hard'' expectation maximization for partially-supervised data with latent variables \cite{bach-inferning13}. In our model, we exploit this to represent student engagement as a latent variable.

\subsection{Features}
\label{sec:features}
Our PSL model is built on a foundation of observable features from the data. In the following, we detail the various features we collect from the data. Following the first-order logic based syntax of PSL, we describe these features as logical predicates.

\paragraph{Behavioral Features}

Our behavioral features are attributes that the learner exhibits while on the MOOC website. In our models, we consider two types of behavioral features: aggregate and non-aggregate. The aggregate features give the overall behavioral metrics, while the non-aggregate features are at the instance level. The predicates \textit{postActivity}(USER) and \textit{voteActivity}(USER) capture how active the user is in the forums. These are calculated for each user by assessing whether the user posts more than the average number of posts generated by all users. Predicate \textit{reputation}(USER) represents whether the overall reputation of a user is above average. Predicate \textit{posts}(USER, POST) and  \textit{votes}(USER, POST) capture an instance-level log of users posting and voting on the discussion forums. Predicate \textit{upvote}(POST) is true if the post has positive votes and false otherwise, and predicate \textit{downvote}(POST) is true if a post has been downvoted. 

\paragraph{Linguistic Features}
The subjectivity and polarity scores of the posts are from \emph{OpinionFinder} \cite{opinionfinder}, represented by \textit{subjective}(POST) and \textit{polarity}(POST) respectively. Both these predicates are calculated by normalizing the number of subjective/objective tags and positive and negative polarity tags marked by opinion finder.

\paragraph{Temporal Features}
The course is divided into the following time periods: one time period before the course starts and three time periods during the course. The time period splits during the course are constructed according to the deadlines in the course---one before the first assignment, second and third being before and after the midterm respectively. The temporal features \textit{lastQuiz}, \textit{lastLecture}, \textit{lastPost}, \textit{lastView} and \textit{lastVote} capture the time-period in which each last interaction of the user occurred. These features measure to what lengths the user participated in different aspects of the course.

\paragraph{Structural Features}
We also include structural features induced by forum structure. These are given by \textit{sameThread}(POST\_1, POST\_2) and \textit{sameForum}(THREAD\_1, THREAD\_2) to capture posts in the same thread and threads in the same forum respectively. Including these relationships allows for our model to capture structural phenomena in interactions and how it can affect student engagement and performance.

\subsection{PSL Models}
\label{sec:pslmodels}
We now construct two different PSL models for predicting learner performance in a MOOC setting---1) simple model that directly infers learner performance from observable features and 2) latent variable model that infers student engagement as a hidden variable to predict learner performance.

In our simple PSL model, we model performance of user by using the observable behavioral features exhibited by the learner, linguistic features corresponding to the content of the post and structural features derived from forum interaction. Meaningful combinations of one or more observable behavioral features described in Section \ref{sec:features} are used to predict \emph{performance}. Table \ref{table:simplepslmodel} gives a subset of rules used in this model (\emph{U} and \emph{P} in tables \ref{table:simplepslmodel} and \ref{table:latentpslmodel} refer to USER and POST respectively). As can be seen, the observable features directly imply performance of the learner in the simple model.

%\begin{align*}
%w_1 : postActivity(USER) \wedge reputation(USER) \rightarrow performance(USER)\\
%w_2 : voteActivity(USER) \wedge reputation(USER) \rightarrow performance(USER) \\
%w_3 : posts(USER, POST) \wedge positive(POST) \rightarrow performance(USER) \\
%w_4 : posts(USER, POST) \wedge subjective(POST) \rightarrow performance(USER)\\
%w_5 : posts(USER, POST) \wedge upvote(POST) \rightarrow performance(USER)\\
%w_6 : posts(USER_1, POST_1) \wedge performance(USER_1) \wedge sameThread(USER_1) \rightarrow performance(USER_2)
%\end{align*}

%\begin{table}
%\tiny{
%\begin{center}
%	\begin{tabular}{ | l | p{8 cm}| l |p{5cm} | }
%    \hline
%    Weights  &  Rules (Model 1) & Rules (Model 2)\\ \hline
%$w_1$ &  $postActivity(USER) \wedge reputation(USER) \rightarrow performance(USER) $ & $ postActivity(USER) \wedge reputation(USER) \rightarrow eActive(USER)$\\
%$w_2$  & $voteActivity(USER) \wedge reputation(USER) \rightarrow performance(USER)$ &  $voteActivity(USER) \wedge reputation(USER) \rightarrow ePassive(USER)$ \\
%$w_3$ & $posts(USER, POST) \wedge positive(POST) \rightarrow performance(USER)$ &  $ posts(USER, POST) \wedge positive(POST) \rightarrow eActive(USER)$ \\
%$w_4$ & $votes(USER, POST) \wedge positive(POST) \rightarrow performance(USER)$ & $votes(USER, POST) \wedge positive(POST) \rightarrow ePassive(USER)$\\
%$w_5$ & $posts(USER, POST) \wedge upvote(POST) \rightarrow performance(USER)$ &  $posts(USER, POST) \wedge upvote(POST) \rightarrow eActive(USER)$\\
%    \hline
%    \end{tabular}
%    \end{center}
%      \caption{PSL models}
%                    %\vspace{-1.0cm}
%  \label{table:notations}
%}
%\end{table}

We can enhance this type of model by adding latent variables that have semantic meaning but can not be directly measured from the data. We treat learner engagement as a latent variable and associate the various observed features to one or more forms of engagement and use this to predict learner performance. The latent variables in this model are represented by predicates \textit{engagement\_active}, \textit{engagement\_passive} and \textit{disengagement}. In this model, some of the observable features---\emph{postActivity}, voteActivity, viewActivity are used to classify learners into one or more forms of engagement or disengagement. Then the engagement predicates---\textit{engagement\_active}, \textit{engagement\_passive} and \textit{disengagement} conjuncted with features that were not used in classifying user engagement like \emph{reputation} are then used to predict learner performance. Some rules from this model are given in Table \ref{table:latentpslmodel}.

%\textbf{Bert: This model description seems incomplete. What are the rules that imply performance? Are there constraints on the latent variables now? Both cases need pointers from the text to the tables, or if they are just equation blocks that's not necessary.}
%
%\textbf{Bert: I tried to clean things up, but the model description is odd. There's no need for a table, because adding the weight value doesn't add anything to the discussion. We can just say in text that each rule has its own weight. So I would just type these as a block of equations, like you have in the comments}

Learning a model with latent engagement can suggest which forms of engagement are good indicators of learner performance. The weights of the model are learned by performing hard expectation-maximization with performance as a target variable. The resulting model results in better predictive performance and can provide more insight into MOOC user behavior than a simpler model.

\begin{table}
%\begin{center}
\parbox{.45\linewidth}{
\vspace{1.22cm}
	\begin{tabular}{ ll  }
    \hline
\tiny{$\textit{postActivity}(U) \wedge \textit{reputation}(U) \rightarrow \textit{performance}(U) $ }\\
\tiny{$\textit{voteActivity}(U) \wedge \textit{reputation}[(U) \rightarrow \textit{performance}(U)$ }\\
\tiny{$\textit{posts}(U, P) \wedge \textit{positive}(P) \rightarrow \textit{performance}(U)$ }\\
\tiny{$\textit{votes}(U, P) \wedge \textit{positive}(P) \rightarrow \textit{performance}(U)$ }\\
\tiny{$\textit{posts}(U, P) \wedge \textit{upvote}(P) \rightarrow \textit{performance}(U)$ }\\
    \hline
    \end{tabular}
      \caption{Rules for simple PSL model. }
                    %\vspace{-1.0cm}
  \label{table:simplepslmodel}
}
\parbox{.55\linewidth}{
	\begin{tabular}{ c }
    \hline
\tiny{$ \textit{postActivity}(U) \wedge \textit{reputation}(U) \rightarrow \textit{eActive}(U)$}\\
\tiny{$\textit{voteActivity}(U) \wedge \textit{reputation}(U) \rightarrow \textit{ePassive}(U)$} \\
\tiny{$ \textit{posts}(U, P) \wedge \textit{positive}(P) \rightarrow \textit{eActive}(U)$ }\\
\tiny{$\textit{votes}(U, P) \wedge \textit{positive}(P) \rightarrow \textit{ePassive}(U)$}\\
\tiny{$\textit{posts}(U, P) \wedge \textit{upvote}(P) \rightarrow \textit{eActive}(U)$}\\
\tiny{$\textit{posts}(U_1, P_1) \wedge \textit{posts}(U_2, P_2) \wedge \textit{eActive}(U_1) \wedge \textit{sameThread}(P_1, P_2)$}\\
\tiny{$\rightarrow \textit{eActive}(U_2)$}\\
\tiny{$\textit{eActive}(U) \wedge \textit{ePassive}(P) \rightarrow \textit{performance}(U)$}\\
    \hline
    \end{tabular}
      \caption{\noindent Rules for PSL model with latent variables.}
                    %\vspace{-1.0cm}
  \label{table:latentpslmodel}
}
 %   \end{center}
\end{table}


%\subsection{Model 3}
%In this model, we also take into account the temporal features and capture the observed variables over the various time periods in the course and connections between the variables across time periods.  To do so, the course is partitioned into the following time periods - 1) before course starts, 2) first month of course 3) second month of course and 4) after course ends.
%
%Here we make use of latent variables for engagement and connect them across time periods to see how learners transition between one form of engagement to another and see how that affects course completion.
%
%TODO: rules from white board
%The rules in model 3 are changed to include the latent variable engagement.
%\begin{align*}
%postActivity(USER) \wedge reputation(USER) \rightarrow eActive(USER)\\
%voteActivity(USER) \wedge reputation(USER) \rightarrow ePassive(USER)\\
%posts(USER, POST) \wedge positive(POST) \rightarrow eActive(USER)\\
%votes(USER, POST) \wedge positive(POST) \rightarrow ePassive(USER)\\
%posts(USER, POST) \wedge upvote(POST) \rightarrow eActive(USER)\\
%posts(USER_1, POST_1) \wedge eActive(USER_1) \wedge sameThread(POST_1, POST_2) \rightarrow eActive(USER_2)
%\end{align*}