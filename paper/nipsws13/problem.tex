\section{Problem Statement}
\label{sec:problem}

Decreasing the student dropout rate is arguably the biggest challenge faced by instructors conducting online courses. Recent studies \cite{kuh}  show that only about 5 percent of students who register for the course eventually end up completing it and obtaining a grade in it. This proportion is also consistent with the data we used (collected from the ``Surviving Disruptive Technologies" course, described in Section \ref{sec:experiments}), in which only 5.11\% of the students completed the course.
 
In this work we take a step towards helping educators in this goal using data driven methods. We analyze the learners on-line behavior in an attempt to identify how students engage with the course materials, and how this engagement can help predicting if the learner will successfully complete the course. We follow the intuition that level and type of student engagement with in the course provides good indication, and distinguish between different types of engagement and how they relate to different activity patterns. For example, we found that learners who do not take quizzes during lectures still follow forum discussion, indicating a passive form of engagement. For these users engagement is typically manifested as viewing and voting forum activities. 
In this paper we formalize this intuition, and construct a probabilistic model capturing the latent engagement level of students.

%
%Our work is motivated by answering three questions- 
%
%{\it i) How to model learner engagement in MOOCs?}
%
%{\it ii) How to leverage learner engagement to predict performance (in this case grades)?}
%
%{\it iii) What are the key indicators that suggest course completion and performance of learners?}
